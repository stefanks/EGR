\documentclass{article}
\usepackage{amsmath, fullpage}

\newcommand{\grad}{\nabla}

\begin{document}
 
\noindent 
Let us consider the following matrix
\[
 M_k = \begin{pmatrix} (1-\eta)(I+\alpha H_k) & \displaystyle\frac{1}{L} (1-\eta)\alpha H_k\bar H_{k-1}\\  
                   -\alpha LI  & I-\alpha\bar H_{k-1} \end{pmatrix}                  
\]
with $\alpha=\displaystyle\frac{\eta}{4L}$.

\bigskip
\noindent
In our experiments on the minimization of the sum of random quadratics, we observed that this matrix always have all eigenvalues less than 1 (even if it has singular values slightly larger than 1!).  As expected, the maximum eigenvalue decreases as $\eta$ increases (which should indicate a higher convergence rate).  The following table illustrates the situation on instances with problem dimension($n$)=20, and component functions($N$)=100.  In all cases, the condition number is around $10^{-2}$, and $L$ is around $5\times 10^3$.
\begin{center}
\begin{tabular}{ccc}
 \hline
 $\eta$ & max(eigenvalue(M\_k)) & max(singular\_value(M\_k)) \\
 \hline
 0.01 & 0.9999 & 1.0001 \\
 0.05 & 0.9998 & 1.0006 \\
 0.1 & 0.9997 & 1.0013 \\
 0.5 & 0.9983 & 1.0087 \\
 \hline
\end{tabular}
\end{center}

\bigskip

If we can generalize our numerical observation and prove that the spectral radius of this matrix is always less than 1, that would indicate 
\[
\prod_{t=1}^\infty M_t \rightarrow 0. 
\]
Since the random processes at each iteration are independent, this would provide $E[e_k, x_k-x_\ast]\rightarrow 0$ (did not yet think about the variance).


\bigskip

\paragraph{Bounding the spectral radius.}
To simplify $M_k$, let us for now set $\bar H_{k-1}=H_k=H$.  So, $M_k$ becomes
\[
M = \begin{pmatrix} (1-\eta)(I+\displaystyle\frac{\eta}{4L} H) & \displaystyle(1-\eta)\frac{\eta}{4L^2} HH\\ 
 \\
                   -\displaystyle\frac{\eta}{4}I  & I-\displaystyle\frac{\eta}{4L} H \end{pmatrix}.                  
\]
Let us first observe that 
\begin{align*}
\sum_i \lambda_i = trace(M) &= trace(M_{11})+trace(M_{22})\\
&=trace((1-\eta)(I+\frac{\eta}{4L} H))+trace(I-\frac{\eta}{4L} H)\\
&\geq n(1-\eta)(1+\frac{\eta\mu}{4L})+n(1-\frac{\eta}{4})>0,
\end{align*}
and
\begin{align*}
\prod_i \lambda_i = \det(M) &= \det\left((1-\eta)(I+\alpha H)(I-\alpha H) + (1-\eta)\alpha^2 HH\right)\\
&=\det\left((1-\eta)I\right) = (1-\eta)^n<1.\\
\end{align*}

\bigskip
\paragraph{Lemma 1.} The largest eigenvalue of $M$ is less than 1.  

\bigskip

\noindent
\textit{Proof.} Each eigenvalue $\lambda$ of $M$ by definition satisfies $\det(M-\lambda I)=0$.  Therefore,
\[
 \det \begin{pmatrix} (1-\eta)(I+\displaystyle\frac{\eta}{4L} H)-\lambda I & \displaystyle(1-\eta)\frac{\eta}{4L^2} HH\\ 
 \\
                   -\displaystyle\frac{\eta}{4}I  & I-\displaystyle\frac{\eta}{4L} H -\lambda I  \end{pmatrix}=0
\]
Since blocks $M_{21}$ and $M_{22}$ commute we have
\[
 \det\left((1-\eta)(I+\frac{\eta}{4L} H)-\lambda I)(I-\frac{\eta}{4L} H -\lambda I) + (1-\eta)\frac{\eta^2}{4^2L^2} HH \right)=0.
\]

With some algebra, this statement simplifies to 
\[
 \det\left((1-\lambda)(1-\lambda-\eta)I + \eta^2 \frac{\lambda}{4L}H \right)=0.
\]

Then, $-(1-\lambda)(1-\lambda-\eta)$ is an eigenvalue of the matrix $\eta^2 \displaystyle\frac{\lambda}{4L}H$; so,
\begin{equation}
 -(1-\lambda)(1-\lambda-\eta) \in \left[\frac{\eta^2\lambda\mu}{4L},\frac{\eta^2\lambda}{4}\right]
 \label{eq:p}
\end{equation}
should hold.

The concave function $p(\lambda)= -(1-\lambda)(1-\lambda-\eta)$ has two roots at $\lambda=1$ and at $\lambda=1-\eta$.  Then, $p(\lambda)<0$ for $\lambda>1$ and for $\lambda<1-\eta$.  So, having $\lambda>1$ is inconsistent with \eqref{eq:p}.  

\bigskip

\paragraph{Lemma 2.}$M$ has no negative eigenvalues.  

\bigskip

\noindent
\textit{Proof.} Suppose that $\lambda<0$ is an eigenvalue of $M$.  Then, $p(\lambda)\geq \displaystyle\frac{\eta^2 \lambda}{4}$ by \eqref{eq:p}(since $\lambda<0$). That is,
\begin{align*}
 &-(1-\lambda)(1-\lambda-\eta) \geq \displaystyle\frac{\eta^2 \lambda}{4}\\
 \Rightarrow&-\lambda^2+(2-\eta-\displaystyle\frac{\eta^2}{4})\lambda+\eta-1\geq 0.  
\end{align*}
The smaller root of the polynomial on the left hand side is
\begin{align*}
 &-\frac{1}{2}\left(-(2-\eta-\displaystyle\frac{\eta^2}{4})+\sqrt{(2-\eta-\displaystyle\frac{\eta^2}{4})^2+4(1-\eta)}\right)\\
 &=1-\frac{\eta}{2}-\frac{\eta^2}{8}-\frac{1}{8}\sqrt{\eta^4+8\eta^3}>1-\eta\geq 0 \quad \mbox{for}\quad \eta\in(0,1]
\end{align*}
as $\ \displaystyle\frac{\eta}{2}+\frac{\eta^2}{8}+\frac{1}{8}\sqrt{\eta^4+8\eta^3}<\frac{\eta}{2}+\frac{\eta}{8}+\frac{3\eta}{8}=\eta$.
Therefore, $p(\lambda)\geq \displaystyle\frac{\eta^2 \lambda}{4}$ cannot be satisfied for $\lambda<0$.  

\bigskip

Thus, we can conclude that for all eigenvalues of $M$ we have $\lambda\in(1-\eta,1)$.  We will next investigate the $\lambda$ values that satisfy $p(\lambda)\leq \displaystyle\frac{\eta^2 \lambda}{4}$ and $p(\lambda)\geq \displaystyle\frac{\eta^2 \lambda\mu}{4L}$ to obtain a tighter bound for the spectral radius.  Even if we saw that $\lambda$ is strictly less than 1 provided that we have strong convexity ($\mu>0$), we wish to know how it depends on $\mu$, $L$, and $\eta$.

\bigskip

\paragraph{Theorem.} For all eigenvalues $\lambda$ of $M$
\[
 \lambda\in\left(1-\eta \ , \ 1-\frac{\eta}{2}-\frac{\eta^2}{2}\right)\cup\left(1-\frac{\eta}{2}+\frac{\eta^2}{4} \ , \ 1-\frac{\mu\eta^2}{4L}\right)
\]
holds.  So, the spectral radius of $M$ is less than $1-\displaystyle\frac{\mu\eta^2}{4L}$.


\bigskip

\noindent
\textit{Proof.} 
\begin{itemize}
 \item[I.] Consider $p(\lambda)\leq \displaystyle\frac{\eta^2 \lambda}{4}$. We have already derived the roots of the polynomial $p(\lambda)-\displaystyle\frac{\eta^2 \lambda}{4}$. Then, 
 for the smaller root $\lambda_{s1}$ we have $\lambda_{s1}<1-\displaystyle\frac{\eta}{2}-\displaystyle\frac{\eta^2}{2}$, and for the larger root $\lambda_{l1}$ we have $\lambda_{l1}>1-\displaystyle\frac{\eta}{2}+\displaystyle\frac{\eta^2}{4}$.  So, if $p(\lambda)-\displaystyle\frac{\eta^2 \lambda}{4}\leq 0$, then either $\lambda < 1-\displaystyle\frac{\eta}{2}-\displaystyle\frac{\eta^2}{2}$ or $\lambda > 1-\displaystyle\frac{\eta}{2}+\displaystyle\frac{\eta^2}{4}$ (but not vice versa).
 \item[II.] Consider $p(\lambda)\geq \displaystyle\frac{\eta^2 \lambda\mu}{4L}$.  The larger root of the polynomial $p(\lambda)-\displaystyle\frac{\eta^2 \lambda\mu}{4L}$ is
 \[
  \lambda_{l2} = 1-\frac{\eta}{2}-\frac{\mu\eta^2}{8L}+\frac{\eta}{2}\sqrt{1-\frac{\mu}{L}+\frac{\mu\eta}{2L}+\left(\frac{\mu\eta}{4L}\right)^2}<1-\frac{\mu\eta^2}{4L}
 \]
as 
 \[
  \sqrt{1-\frac{\mu}{L}+\frac{\mu\eta}{2L}+\left(\frac{\mu\eta}{4L}\right)^2} = \sqrt{\left(1-\frac{\mu\eta}{4L}\right)^2-(1-\eta)\frac{\mu}{L}}<1-\frac{\mu\eta}{4L}.
 \]
Using the same argument, we observe that for the smaller root we have $\lambda_{s2}>1-\eta$.  So, if $p(\lambda)- \displaystyle\frac{\eta^2 \lambda\mu}{4L}\geq 0$, then $\lambda\in(1-\eta,1-\displaystyle\frac{\mu\eta^2}{4L})$ (but not vice versa). Merging all the bounds we get the result.

 
\end{itemize}


% \newpage
% 
% \paragraph{An earlier attempt to get a bound.}
% Let us first check the determinant of $M_k$ (product of all eigenvalues).  A result by Silvester (see Theorem 3 in paper \emph{block\_det.pdf} in this folder) simplifies the computation as follows.
% \begin{align*}
% \det(M_k) &= \det\left((1-\eta)(I+\alpha H_k)(I-\alpha\bar H_{k-1}) + (1-\eta)\alpha^2 H_k\bar H_{k-1}\right)\\
% &=\det\left((1-\eta)(I+\alpha(H_k-\bar H_{k-1}))\right) = (1-\eta)^n\det\left(I+\alpha(H_k-\bar H_{k-1})\right)\\
% &\leq (1-\eta^2)^n<1
% \end{align*}
% by Hadamard's inequality provided that $\max(diag(H_k-\bar H_{k-1}))\leq 4L$ holds and the matrix $(I+\alpha(H_k-\bar H_{k-1}))$ is positive definite. (For the quadratic case, $H_k=\bar H_{k-1}$, therefore $\det(M_k)=(1-\eta)^n$).
% 
% \bigskip
% 
% Now, we will try to provide a bound on the spectral radius of $M_k$ using Theorem 5 in paper \emph{spectral\_bound.pdf} by Derzko and Pfeffer, which provides a bound based on the determinant (in this folder as well).
% 
% \bigskip
% 
% (I quitted this approach because I numerically tested the bound in the above paper and saw that it does not provide a tight enough bound. I tried some other bounds that I found in other papers but none were tight enough to provide a bound that is less than 1. So I decided to directly work on the value of the maximum eigenvalue rather than a bound on it).


\end{document}

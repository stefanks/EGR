\documentclass{article}
\usepackage{amsmath}
\usepackage{amsthm}
\usepackage{amsfonts}
\usepackage[margin=0in, landscape]{geometry}
\usepackage[usenames,dvipsnames]{pstricks}
\usepackage{epsfig}
\usepackage{epstopdf}
\usepackage{pst-grad}
\usepackage{pst-plot}
\usepackage{fancyhdr}
\usepackage{amsthm}
\pagestyle{fancy}
\fancyhead{}
\fancyfoot{} 
\lhead{Stefan\\ \large Results }
\rfoot{\thepage}

\newtheorem{defn}{Definition}[section]
\newtheorem{thm}{Theorem}[section]
\newtheorem{prop}{Proposition}


\begin{document}

\section{Feb 16th}

I ran the SQN and SGD algorithms on speech and yoram problems. 
Tested batch sizes $1,2,5,10,50$. The options for SGD are:
\begin{verbatim}
opts = 
            stepsizeinit: 0.1250
            stepsizepowk: 1
            useAveraging: 0
                maxiters: 287412
\end{verbatim}
The options for SQN are:
\begin{verbatim}
opts = 
                       L: 200
                       M: 5
        useLimitedMemory: 1
            stepsizeinit: 0.1250
            stepsizepowk: 1
        hessianBatchSize: 1000
                maxiters: 47902
        startQuasiNewton: 0
            useAnchorAvg: 0
\end{verbatim}
\subsection{speech problem}
\includegraphics[scale=1]{figures/speechfeb16.eps} 
\subsection{yoram problem}
\includegraphics[scale=1]{figures/yoramfeb16.eps} 

\subsection{Increasing initial stepsize}
Same parameters as in the beginning of the section, except stepsizeinit = 1.

\includegraphics[scale=1]{figures/yoramfeb16_bis1Lis200.eps} 

\subsection{Decreasing L}
Now L is 20, bh is 100
\begin{verbatim}
                       L: 20
                       M: 5
        useLimitedMemory: 1
            stepsizeinit: 0.1250
            stepsizepowk: 1
        hessianBatchSize: 100
             outputLevel: 3
              globalSeed: 1
       gradientBatchSize: 50
                maxiters: 182
        cheapOutputEvery: 1
    expensiveOutputEvery: 10
        startQuasiNewton: 0
            useAnchorAvg: 0
\end{verbatim}
For yoram problem:

\includegraphics[scale=1]{figures/yoramfeb16_Lis20.eps} 

\subsubsection{Running yoram for 20 passes over data}
For sqn:
\begin{verbatim}
opts = 
                       L: 20
                       M: 5
        useLimitedMemory: 1
            stepsizeinit: 0.1250
            stepsizepowk: 1
        hessianBatchSize: 100
             outputLevel: 3
              globalSeed: 1
       gradientBatchSize: 10
                maxiters: 6667
        cheapOutputEvery: 66
    expensiveOutputEvery: 10
        startQuasiNewton: 0
            useAnchorAvg: 0
\end{verbatim}
\includegraphics[scale=1]{figures/yoramfeb16_20passes.eps}


\subsection{trying larger initial stepsize}
For sgd:
\begin{verbatim}
opts = 
            stepsizeinit: 1
            stepsizepowk: 1
            useAveraging: 0
             outputLevel: 3
              globalSeed: 1
               batchSize: 50
                maxiters: 2000
        cheapOutputEvery: 20
    expensiveOutputEvery: 10
\end{verbatim}
For sqn:
\begin{verbatim}
                       L: 20
                       M: 5
        useLimitedMemory: 1
            stepsizeinit: 1
            stepsizepowk: 1
        hessianBatchSize: 100
       gradientBatchSize: 5
                maxiters: 10000
\end{verbatim}
\includegraphics[scale=1]{figures/yoramfeb16_bis1.eps} 

\subsection{RCV1}
Options for GG:
\begin{verbatim}
            stepsizeinit: 1
            stepsizepowk: 1
            useAveraging: 0
               batchSize: 1
                maxiters: 1032494
\end{verbatim}
Options for SQN:
\begin{verbatim}
opts = 
                       L: 200
                       M: 5
        useLimitedMemory: 1
            stepsizeinit: 1
            stepsizepowk: 1
        hessianBatchSize: 1000
       gradientBatchSize: 50
                maxiters: 18773
\end{verbatim}
\includegraphics[scale=1]{figures/RCV1feb16.eps} 

\newpage

\section{Feb 17th}
\subsection{yoram}
I ran the yoram problems with amount of work equivalent to 40 passes over data in sgd. The options are:
For gd:
\begin{verbatim}
    stepsizeinit: 1
    stepsizepowk: 1
    useAveraging: 0
\end{verbatim}
For sqn:
\begin{verbatim}
                    L: 200
                    M: 5
     useLimitedMemory: 1
         stepsizeinit: 1
         stepsizepowk: 1
     hessianBatchSize: 1000
\end{verbatim}
\newpage

\includegraphics[scale=0.95]{figures/Feb17-yoramiters.eps}\includegraphics[scale=0.95]{figures/Feb17-yoramadp.eps} 

\includegraphics[scale=0.95]{figures/Feb17-yoramwork.eps} 

\subsection{speech}
I ran speech for work equivalent to 5 passes over data.
sqn options:
\begin{verbatim}
                    L: 200
                    M: 5
     useLimitedMemory: 1
         stepsizeinit: 1
         stepsizepowk: 1
     hessianBatchSize: 1000
     startQuasiNewton: 0
         useAnchorAvg: 0
\end{verbatim}
options for gd:
\begin{verbatim}
    stepsizeinit: 1
    stepsizepowk: 1
    useAveraging: 0
\end{verbatim}
\newpage

\includegraphics[scale=0.95]{figures/Feb18-speechiters.eps} 
\includegraphics[scale=0.95]{figures/Feb18-speechapd.eps} 


\includegraphics[scale=0.95]{figures/Feb18-speechwork.eps} 

\subsection{rcv1}
\newpage

\includegraphics[scale=0.95]{figures/Feb18-rcv1iters.eps} 
\includegraphics[scale=0.95]{figures/Feb18-rcv1adp.eps} 


\includegraphics[scale=0.95]{figures/Feb18-rcv1work.eps} 

\end{document}
